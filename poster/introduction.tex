\documentclass[letterpaper]{amsart}

\usepackage{amsmath}
\usepackage{amssymb}
\usepackage{amsthm}
\usepackage{fullpage}
\usepackage{enumerate}
\usepackage{url}
\usepackage{graphicx}
\usepackage{hyperref}

\renewcommand*\familydefault{\sfdefault}

\newcommand{\vectornorm}[1]{\left\lVert#1\right\rVert}
\newcommand{\abs}[1]{\left\lvert#1\right\rvert}
\newcommand{\floor}[1]{\left\lfloor#1\right\rfloor}

\setlength\parindent{0pt}
\setlength\parskip{0.2in}

\thispagestyle{empty}

\begin{document}
\begin{center}
\Huge
Introduction
\end{center}
\huge

Most folk dance music contains two musical themes; a tune begins with one theme (the $A$ section) which is usually repeated, then progresses to another theme (the $B$ section) with similar musical structure and motives.

Goal: understand at a quantitative level the relationship between $A$ and $B$ sections.
\begin{itemize}
\item In a broad sense, what are $A$ sections like? What are $B$ sections like? (SVM classification experiments).

\item How are the $A$ and $B$ sections of a particular tune related? \\(Metric learning).
\end{itemize}

Dataset: The Session $[2]$ makes available a database of roughly 21 thousand such dance tune settings in ABC notation, a human-readable symbolic music data format in plain text.

\end{document}
